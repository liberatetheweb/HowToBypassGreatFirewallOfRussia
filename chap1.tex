\chapter{Введение}
Россия долго воздерживалась от введения централизованного черного списка сайтов, ограничиваясь точечными блокировками сайтов по искам к провайдеру и слежкой за пользователями с использованием СОРМ. Однако 28 июля 2012 года был принят Федеральный закон № 139-ФЗ, согласно которому с 1 ноября 2012 года начал действовать Единый реестр запрещенных сайтов. Для блокировки ресурсов на территории страны (иногда и за ее пределами, из-за аренды иностранными компаниями линий у российских компаний) теперь даже не нужно решение суда. Блокировка зачастую осуществляется по IP-адресу, из-за чего одна запись в реестре может приводить к блокированию тысяч сайтов. Эксперты, отвечающие за вынесение решения о включении ресурса в реестр, зачастую выносят просто абсурдные решения. Например, на территории России запрещено видео с инструкцией по нанесению зомби-макияжа, причем Youtube проиграла иск к Роскомнадзору об удалении информации об этом видео из реестра\cite{zombie}. Была запрещена страница, описывающее применение одного из элементов онлайн-игры EVE Online, на сленге игроков называемого <<наркотиками>>\cite{eve}. На некоторое время блокировались такие крупные сайты, как Google\cite{google}, Яндекс\cite{yandex}, Вконтакте\cite{vk}. И таких примеров множество, это не единичные случаи.\\
2 июля 2013 года был принят Федеральный закон № 187-ФЗ, направленный на блокировку ресурсов, распространяющих пиратские фильмы или даже ссылки на них. Этот закон начал действовать 1 августа 2013 года.\\
Учитывая все вышесказанное, в России становится очень сложно пользоваться интернетом даже обычному человеку. С целью помочь именно этому обычному человеку обойти интернет-цензуру и была написана эта книга.
