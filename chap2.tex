\chapter{Уровень интернет-цензуры в России}
С 2008 по 2012 год правозащитной организацией <<Агора>> было зафиксировано 72 попытки введения регулирования Интернета, 16 нападений на блогеров, 160 уголовных преследований, 56 гражданско-правовых санкций, 386 фактов административного давления, 848 случаев ограничения доступа, 124 случая цензуры\cite{agoranet,agoranet2011,agoranet2012}. По данным Freedom House, по уровню свободы слова в Интернете в 2012 году Россия занимала 30 место из 47\cite{netfreedom}. Организация <<Репортеры без границ>> в 2012 году отнесла Россию к списку стран <<под наблюдением>>\cite{rsf}. Open Net Initiative в 2010 году отметил в России наличие выборочной политической и социальной интернет-цензуры\cite{opennet}.\\
Многие эксперты посчитали Федеральный закон № 139-ФЗ от 28 июля 2012 года, согласно которому был создан Единый реестр запрещенных сайтов, недоработанным и обратили внимание на возможность использовать его для интернет-цензуры\cite{lenta}. В частности, против него выступили Совет по правам человека при президенте РФ\cite{139_presidentsoviet}, Русская Википедия\cite{139_wiki}, Яндекс\cite{139_yandex}, Google\cite{139_google}, LiveJournal\cite{139_livejournal}, Вконтакте\cite{139_vk}.\\
Согласно Федеральному закону № 187-ФЗ от 2 июля 2013 года, также известному как <<антипиратский закон>>, создается аналогичный реестр для страниц, на которых размещается нелицензионное видео, причем ресурс может быть заблокирован даже за ссылку на такой контент. Против этого закона выступили Google, Яндекс, Mail.ru Group, Афиша-Рамблер-СУП, RU-CENTER, Хостинг-Центр, Фонд содействия развитию технологий и инфраструктуры Интернета, Викимедиа РУ, Ozon.ru, Российская Ассоциация электронных коммуникаций, Ассоциация интернет-издателей\cite{187_raec}, Флибуста, КулЛиб, Максима\cite{187_lib}, Look At Media\cite{187_look} и многие другие. 1 августа 2013 года прошла интернет-забастовка, в которой приняли участие около 1600 сайтов\cite{187_strike}. Петиция на сайте Российской общественной инициативы с требованием отменить этот закон набрала более 100 000 подписей\cite{187_roi}. 
