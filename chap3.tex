\chapter{Как работает интернет-цензура}
\epigraph{Поэтому и заложили, что и IP-адрес, не только доменное имя, URL-ы. Потому что IP-адрес --- это совершенно точная адресация, именно поэтому.}{Елена Мизулина \cite{mizulina}}
Российский закон о цензуре (ФЗ-139 от 28 июля 2012 года) подробно не описывает технические требования к системе цензуры, оставляя выбор методов на провайдера. Роскомнадзор выпустил доклад <<Анализ существующих методов управления доступом к интернет-ресурсам и рекомендации по их применению>>, в котором предложил использовать блокировку по URL после фильтрации запросов по IP-адресам. Также было предложено использовать блокировку доменов на уровне авторитативных DNS-серверов в качестве дополнительного метода\cite{rsoc}. Ниже перечислены основные методы, используемые в России и других странах.
\section{Блокировка домена на DNS-серверах провайдера}
\textbf{DNS} --- система доменных имен, позволяющая обращаться к серверу по доменному имени вместо IP-адреса.\\
Чаще всего провайдеры имеют собственные DNS-сервера, что позволят им отдавать вместо правильного IP-адреса любой другой или не отдавать ничего.
\section{Блокировка по IP-адресу}
Провайдер имеет возможность заблокировать IP-адреса, принадлежащие запрещенному ресурсу. Этот способ довольно часто используется в России. Так как на одном IP-адресе могут работать несколько сайтов, то обычно заблокированными оказываются и невинные сайты, что было отражено в исследовании РосКомСвободы: \url{http://reestr.rublacklist.net/}.
\section{Deep Packet Inspection}
DPI позволяет заглядывать внутрь пакетов и фильтровать их по содержимому. Этот способ требует установки довольно дорогого оборудования и позволяет блокировать как конкретные URL, так и саму запрещенную информацию вместе со всеми зеркалами, или даже целые протоколы (например, VPN).
