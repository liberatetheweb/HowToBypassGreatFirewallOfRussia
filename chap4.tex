\chapter{Методы обхода интернет-цензуры}
\begin{important}
Методы, описанные в данной главе, не направлены на обечение полной анонимности и безопасности, они лишь позволяют обойти цензуру.
\end{important}
\section{NoZapret}
\textbf{NoZapret} --- специальное расширение для браузеров Firefox и Chrome/Chromium, пропускающее через прокси-сервера те сайты, которые блокируются на территории России. Сайт --- \url{http://nozapret.org/ext}. Просто установите его, никаких настроек не потребуется.
\section{Использование альтернативных DNS-серверов}
В случае блокировки на уровне DNS-сервера, ее обход возможен с помощью использования альтернативных DNS-серверов. Просто установите в настройках вашего подключения один из следующих адресов.
\subsection{Настройка в Windows}
Откройте \textbf{Панель управления} \textrightarrow \textbf{Сеть и Интернет} \textrightarrow \textbf{Центр сети и общего доступа} \textrightarrow \textbf{Управление сетевыми подключениями}. Наведите курсор на свое подключение, кликните правой кнопкой и выберите пункт \textbf{Свойства}. Далее \textbf{Сеть} \textrightarrow \textbf{Протокол Интернета версии 4 (TCP/IPv4)} \textrightarrow \textbf{Свойства} \textrightarrow \textbf{Использовать следующие адреса DNS-серверов} и введите адреса альтернативных DNS-серверов, после чего нажмите \textbf{ОК}.
\subsection{Настройка в GNU/Linux (NetworkManager)}
Кликните правой кнопкой по значку в трее \textrightarrow \textbf{Изменить соединения}. Выберите свое подключение, нажмите \textbf{Изменить}. Перейдите на вкладку \textbf{Параметры IPv4} и введите адреса DNS-серверов.
\subsection{Настройка в OS X}
Откройте меню Apple \textrightarrow \textbf{Системные настройки} \textrightarrow \textbf{Вид} \textrightarrow \textbf{Сеть}, далее выберите свое подключение и введите адреса DNS-серверов в поле \textbf{DNS-сервер} через запятую.
\subsection{Список альтернативных DNS-серверов}
\begin{enumerate}
\item \url{http://www.opennicproject.org/nearest-servers/} (ближайшие сервера OpenNIC)
\item 8.8.8.8 и 8.8.4.4 (Google Public DNS)
\item 77.88.8.8 (Яндекс.DNS)
\end{enumerate}
\section{Прокси-сервера}
\textbf{Прокси-сервер} --- сервер, который позволяет пропускать через себя пользовательский трафик.
\subsection{Настройка в Firefox}
Откройте \textbf{Настройки} \textrightarrow \textbf{Дополнительные} \textrightarrow \textbf{Сеть} \textrightarrow \textbf{Настроить}. Выберите \textbf{Ручная настройка прокси} и введите адрес и порт прокси, после чего нажмите \textbf{ОК}.
\subsection{Списки открытых прокси}
\begin{enumerate}
\item \url{http://xroxy.com}
\item \url{https://hidemyass.com/proxy-list}
\item \url{http://freeproxy.ch}
\item \url{http://proxylists.net}
\item \url{http://nntime.com}
\end{enumerate}
\section{Веб-прокси}
\textbf{Веб-прокси} --- веб-страница, которая позволяет пользователю получить контент с заданного адреса через себя.
\subsection{Использование}
В случае использования веб-прокси, перейдите на его страницу и введите адрес сайта, который вы хотете посетить.
\subsection{Некоторые веб-прокси}
\begin{enumerate}
\item \url{https://hidemyass.com}
\item \url{http://anonymouse.org}
\item \url{http://hide.pl}
\item \url{http://hideme.ru}
\item \url{http://guardster.com/free/}
\end{enumerate}
\section{VPN}
\textbf{VPN} --- технология, позволяющая создавать сети поверх существующего Интернет-подключения. Из-за высокой скорости работы, простоты настройки и шифрования трафика от клиента до VPN-провайдера часто используется как средство сокрытия реального IP-адреса при доступе в Интернет. VPN-провайдеры обычно предоставляют свои услуги на платной основе, однако вы можете настроить VPN и на своем собственном сервере.
\subsection{Использование}
При использовании VPN, следуйте инструкциям, полученным от вашего VPN-провайдера.
\subsection{Бесплатный VPN}
\begin{enumerate}
\item \url{http://reestra.net}
\item \url{http://kebrum.com}
\end{enumerate}
\subsection{Некоторые VPN-провайдеры}
\begin{enumerate}
\item \url{https://ipredator.se}
\item \url{https://kebrum.com}
\item \url{https://relakks.com}
\item \url{https://vpntunnel.se}
\item \url{http://ivacy.com}
\end{enumerate}
\section{SSH-туннели}
\textbf{SSH} --- протокол, созданный для безопасной передачи данных. Часто используется для удаленного управления другими компьютерами, но может использоваться и для создания туннелей.\\
\textbf{SSH-туннель} --- туннель, созданный с помощью SSH-соединения и используемый для передачи данных. Существуют организации, предоставляющие SSH-туннелирование на платной основе.
\subsection{Использование}
SSH-туннели настраиваются так:
\begin{lstlisting}
ssh -D localhost:port login@address
\end{lstlisting}
port --- порт, трафик на который будет пропускаться через SSH-туннель.\\
login --- ваш логин на удаленном сервере.\\
address --- адрес удаленного сервера.\\
После этого установите в приложениях, трафик которых вы хотите туннелировать, например, в браузере, адрес SOCKS-прокси localhost с портом, который вы указали в предыдущем шаге.
\subsection{Некоторые провайдеры SSH-туннелей}
\begin{enumerate}
\item \url{https://tunnelr.com}
\item \url{http://torvpn.com}
\item \url{http://vpnsecure.me}
\item \url{http://guardster.com}
\item \url{http://anonyproz.com}
\end{enumerate}
\section{I2P}
\textbf{I2P} --- анонимная оверлейная сеть, использующая принцип чесночной маршрутизации, исходные коды которой распространяются на условиях нескольких свободных лицензий. В отличии от Tor, который в первую очередь направлен на доступ к сайтам обычного интернета (хотя в нем и существуют скрытые сервисы, аналогичные ипсайтам в I2P, а в I2P можно получить доступ к внешнему Интернету, используя аутпрокси), основной целью I2P является доступ именно к скрытым ресурсам --- ипсайтам. Ипсайт от обычного вебсайта отличает только его нахождение в сети I2P.
\subsection{Установка}
Для установки посетите \url{http://i2p2.de} или установите пакет с помощью пакетного менеджера вашего дистрибутива.
\subsection{Использование}
После установки настройте свой браузер на использование HTTP-прокси 127.0.0.1:4444 и посетите страницу \url{http://127.0.0.1:7657}. Перед вами консоль маршрутизатора I2P --- место, из которого можно управлять всеми настройками I2P.\\
Для начала перейдите в меню <<Настройки I2P>> (\url{http://127.0.0.1:7657/config}) и установите ограничения скорости в соответствии со скоростью вашего интернета.\\
Остальные настройки можно оставить по умолчанию. Подождите некоторое время для полноценной интеграции с сетью, после чего вы сможете полноценно пользоваться сетью. Роутер желательно не выключать, так как при его перезапуске потребуется придется повторить этот процесс.\\
В I2P отсутствуют корневые DNS-сервера, копия адресной книги хранится на каждом роутере. На \url{http://rus.i2p} вы можете найти дополнительный список подписок, который можно добавить в susidns.
\subsection{Некоторые ипсайты}
\begin{enumerate}
\item \url{http://forum.i2p} --- главный форум, есть рускоязычный раздел.
\item \url{http://rus.i2p} --- русская I2P-вики.
\item \url{http://pastethis.i2p} --- pastebin-подобный ресурс.
\item \url{http://flibusta.i2p} --- зеркало Флибусты, крупной библиотеки электронных книг.
\item \url{http://tracker.rus.i2p} --- русский торрент-трекер в I2P.
\end{enumerate}
\section{Tor}
\textbf{Tor} --- анонимная оверлейная сеть, использующая принцип луковой маршрутизации, исходные коды которой распространяются на условиях лицензии BSD.\\
\textbf{Луковая маршрутизация} --- технология анонимного обмена информацией, использующая многократное шифрование и пересылку через цепочки узлов. Каждый луковый маршрутизатор в цепочке удаляет слой шифрования и пересылает сообщение дальше, согласно полученным инструкциям, где все повторится. И так до тех пор, пока сообщение не достигнет адресата. Такое название технология получила из-за сходства данного процесса с очисткой луковицы.
\subsection{Установка}
Для установки посетите \url{https://torproject.org} или установите пакет с помощью пакетного менеджера вашего дистрибутива.
\subsection{Использование}
Самым простым способом использования Tor является установка Tor Browser Bundle. Просто скачайте его с \url{https://torproject.org/projects/torbrowser.html}, распакуйте и запустите.\\
Вы также можете установить Tor и задать в настройках приложений, которые вы хотите через него использовать, адрес socks5-прокси 127.0.0.1:9050.\\
Для использования приложений через Tor, не имеющих настроек прокси, скачайте torsocks (\url{https://code.google.com/p/torsocks}) или proxychains (\url{http://proxychains.sourceforge.net}).
\subsection{Некоторые скрытые сервисы}
\begin{enumerate}
\item \url{http://dppmfxaacucguzpc.onion} --- TorDir, каталог скрытых сервисов.
\item \url{http://jhiwjjlqpyawmpjx.onion} --- TorMail, электронная почта в Tor.
\item \url{http://silkroadvb5piz3r.onion} --- Silk Road, анонимная торговая площадка, принимающая оплату в Bitcoin.
\item \url{http://4eiruntyxxbgfv7o.onion/pm} --- TorPM, сервис обмена сообщениями.
\end{enumerate}
\section{JonDo}
\textbf{JonDo} (\textbf{JonDonym}, также \textbf{Java Anon Proxy} или \textbf{JAP}) --- программное обеспечение, представляющее доступ к цепочке прокси-серверов.
\subsection{Установка}
Для установки посетите \url{https://anonymous-proxy-servers.net} или установите пакет с помощью пакетного менеджера вашего дистрибутива.
\subsection{Использование}
JonDo напоминает Tor, но в отличии от Tor, где каждый доброволец может поднять как промежуточный сервер, так и exit-ноду, JonDo опирается на помощь отдельных организаций. Однако, Tor может использоваться в цепочке JonDo, для этого достаточно добавить адрес 127.0.0.1:9050 в настройках.\\
Бесплатная версия позволяет проксировать только HTTP и HTTPS трафик, в платной версии доступно все, а также нелимитирована скорость.\\
Для использования запустите JonDo и настройте браузер на использование прокси-сервера 127.0.0.1:4001.
\subsection{Недостатки}
\begin{enumerate}
\item В бесплатной версии можно проксировать только HTTP и HTTPS.
\item В бесплатной версии скорость ограничена до 30--50 кБит/с.
\item В бесплатной версии размер передаваемого файла ограничен 2 МБ.
\item Число нод очень сильно ограничено.
\end{enumerate}
\section{Онлайн RSS-агрегаторы}
Даже если в стране заблокированы какие-то вебсайты, онлайн RSS-агрегаторы позволяют читать RSS-ленты с них, так как отдают ленты не напрямую с запрещенных сайтов, а с собственных серверов.
\subsection{Некоторые RSS-агрегаторы}
\begin{enumerate}
\item \url{https://newsvi.be}
\item \url{http://feedly.com}
\item \url{http://digg.com/reader}
\item \url{http://reader.aol.com}
\item \url{http://theoldreader.com}
\end{enumerate}
